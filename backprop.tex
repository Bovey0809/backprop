\documentclass[12pt, a4paper]{article}

\usepackage[english,ngerman]{babel}
\usepackage{amsmath, amssymb, wasysym}
\usepackage{textcomp}
\usepackage{graphicx}           % Grafikpaket
%\usepackage{graphs}		% c.f. http://www8.cs.umu.se/~drewes/graphs/
\usepackage{tikz}
\usepackage[T1]{fontenc}
% \usepackage[latin9]{inputenc}   % Zeichensatz
\usepackage[utf8]{inputenc}
\usepackage{hyperref}
\usepackage{caption}           % Paket um Bildunterschriften ändern zu können
\usepackage{rotating}           % zum rotieren von Elementen
\usepackage{color}

\usepackage{tikz}

% Abstand nach Bildunterschriften etwas vergrößern
%\addtolength{\belowcaptionskip}{0.2cm}

% Listensymbol einstellen
\renewcommand{\labelitemii}{$\diamond$}

% Format der Bildunterschriften ändern
\renewcommand{\captionfont}{\small\itshape}

% Schriftart setzen
\renewcommand{\familydefault}{\sfdefault}

\definecolor{orange}{rgb}{1,0.5,0}

% Schriftart für Überschriften setzen
%\setkomafont{sectioning}{\bf\rmfamily}

% Seitenränder setzen
\usepackage[top=1cm, left=0.5cm, right=0.5cm, bottom=1.2cm]{geometry}

%% And now it goes looooose :)
\begin{document}

\begin{center}
{\LARGE Herleitung \& Funktionsweise von Backpropagation of Error}
\end{center}

\begin{itemize}

%\item[]
%\underline{Zunächst Nomenklatur des Netzausschnittes:}

\item[]
\underline{Nähere Informationen:}
\\
Hierin wird beschrieben, wie ein künstliches Neuronales Netz des Typs Multi-Layer-Perzeptron mit Backpropagation of Error trainiert wird.
\\
Es wird Grundwissen vorausgesetzt, interessierte Fachfremde verweise ich auf:
\\
\url{http://dkriesel.com/science/neural_networks}.

\item[]
\underline{Netztopologie:}
\def\layersep{2.5cm}

\begin{tikzpicture}[shorten >=1pt,->,draw=black!50, node distance=\layersep]
    \tikzstyle{every pin edge}=[<-,shorten <=1pt]
    \tikzstyle{neuron}=[circle,fill=black!25,minimum size=17pt,inner sep=0pt]
    \tikzstyle{input neuron}=[neuron, fill=green!50];
    \tikzstyle{output neuron}=[neuron, fill=red!50];
    \tikzstyle{hidden neuron}=[neuron, fill=blue!50];
    \tikzstyle{annot} = [text width=4em, text centered]

    % Draw the input layer nodes
    \foreach \name / \y in {1,...,3}
    % This is the same as writing \foreach \name / \y in {1/1,2/2,3/3,4/4}
        \node[input neuron, pin=left:$x_\y$] (I-\name) at (0,-\y) {$\diagup$};

        \node[pin=left:$\vdots$] (I-4) at (0,-4) {};

        \node[input neuron, pin=left:$x_N$] (I-5) at (0,-5) {$\diagup$};

    % Draw the hidden layer nodes
    \foreach \name / \y in {1,...,5}
        \path[yshift=0.5cm]
            node[hidden neuron] (H-\name) at (\layersep,-\y cm) {$\frac{\sim}{\sum}$};

    % Draw the output layer node
    \node[output neuron,pin={[pin edge={->}]right:$y_1$}, right of=H-3] (O) {};

    % Connect every node in the input layer with every node in the
    % hidden layer.
    \foreach \source in {1,...,3}
        \foreach \dest in {1,...,5}
            \path (I-\source) edge (H-\dest);

    \foreach \source in {5,...,5}
        \foreach \dest in {1,...,5}
            \path (I-\source) edge (H-\dest);

    % Connect every node in the hidden layer with the output layer
    \foreach \source in {1,...,5}
        \path (H-\source) edge (O);

    % Annotate the layers
    \node[annot,above of=H-1, node distance=1cm] (hl) {Hidden layer};
    \node[annot,left of=hl] {Input layer};
    \node[annot,right of=hl] {Output layer};
\end{tikzpicture}

\item[]
\underline{Fehlerfunktion:}
\begin{align*}
F &= \sum_{p \in P} \ ^pE \\
^p E &= \frac{1}{2} \sum_{j \in M} \left(\hat{y}_m - y_m \right)^2
\end{align*}

\item[]
\underline{Fehler der Gewichte:}
\begin{align*}
^p \vartriangle w_{hm} &\sim - \nabla_w \cdot \ ^p E \\
\vartriangle w_{hm} &= - \eta \textcolor{red}{\frac{\partial E \left( w_{hm} \right) }{ \partial w_{hm}} }
\end{align*}

\item[]
\underline{Für ein Output-Neuron:}

$net_m = \sum_{i=0}^H w_{im} \tilde{o}_i$

$o_m = y_m = f_m(net_m) $

$ \textcolor{red}{\frac{\partial E \left( w_{hm} \right) }{ \partial w_{hm} }} = \textcolor{magenta}{\frac{\partial E }{ \partial net_m }} \cdot \textcolor{green}{\frac{ \partial net_m }{ \partial w_{hm} }}$

\begin{align*}
\textcolor{green}{\frac{\partial net_m }{ \partial w_{hm} }} &= \frac{ \partial }{ \partial w_{hm} } \cdot net_m \\
&= \frac{ \partial }{ \partial w_{hm} } \sum_{i=0}^H w_{im} \tilde{o}_i \\
&= \sum_{i=0}^H \frac{ \partial }{ \partial w_{hm} } w_{im} \tilde{o}_i \\
&= \frac{ \partial }{ \partial w_{hm} } \tilde{o}_h w_{hm} \\
&= \text{\textcolor{green}{\framebox{$ \tilde{o}_h $}}} \\
\\
\textcolor{magenta}{\frac{\partial E }{ \partial net_m }} &= \frac{ \partial E }{ \partial y_m } \cdot \textcolor{orange}{\frac{ \partial y_m }{ \partial net_m }} \ \left( = \textcolor{magenta}{- \delta_m} \right) \\
&= \frac{ \partial E }{ \partial y_m } \cdot \frac{ \partial }{ \partial net_m } f_m(net_m) \\
&= \underbrace{\textcolor{blue}{\frac{ \partial E }{ \partial y_m }} \cdot \text{\textcolor{orange}{\framebox{$ f'_m (net_m) $}}}}_{=: -\delta_m} \\
\\
\textcolor{blue}{\frac{ \partial E }{ \partial y_m }} &= \frac{ \partial }{ \partial y_m } \cdot \frac{1}{2} \sum_{j=1}^M \left( \hat{y}_j - y_j \right)^2 \\
&= \text{\textcolor{blue}{\framebox{$ - (\hat{y}_m - y_m) $}}} \\
\end{align*}

\begin{align*}
\vartriangle w_{hm} &= - \eta \textcolor{red}{\frac{ \partial E }{ \partial w_{hm} }} \\
&= \textcolor{blue}{-} \eta \textcolor{blue}{\frac{ \partial E }{ \partial y_m }} \textcolor{orange}{\frac{ \partial y_m}{ \partial net_m}} \textcolor{green}{\frac{ \partial net_m}{ \partial w_{hm}}} \\
&= \eta \textcolor{blue}{\left( \hat{y}_m - y_m \right)} \textcolor{orange}{f'_m(net_m)} \textcolor{green}{\tilde{o}_h}
\end{align*}

\framebox{$ \vartriangle w_{hm} = \eta \cdot \delta_m \cdot \tilde{o}_h $}
Widrow-Hoff-Regel / $\delta$-Regel
\\

\item[]
\underline{Für ein Hidden-Neuron:}

$net_h = \sum_{i=0}^H w_{ih} \tilde{o}_i$

$ \textcolor{red}{\frac{\partial E }{ \partial w_{kh} }} = \textcolor{magenta}{\frac{\partial E }{ \partial net_h }} \cdot \textcolor{green}{\frac{ \partial net_h }{ \partial w_{kh} } }$

\begin{align*}
\delta_h &= - \textcolor{magenta}{\frac{ \partial E }{ \partial net_h}} \\
&= \textcolor{blue}{- \frac{ \partial E }{ \partial o_h}} \cdot \textcolor{orange}{\frac{ \partial o_h }{ \partial net_h}} \\
\\
\textcolor{blue}{- \frac{ \partial E }{ \partial o_h}} &= - \frac{ \partial E \left( \underline{net}_{l=1} , \underline{net}_{l=2} , \hdots , \underline{net}_{l=L} \right) }{ \partial o_h} \\
&= \sum_{l=1}^L \left( - \frac{ \partial E}{ \partial \underline{net}_l } \right) \cdot \frac{ \partial \underline{net}_l }{ \partial o_h} \\
&= \sum_{l=1}^L \underline{\delta}_l \cdot \frac{ \partial }{ \partial o_h} \sum_{j=0}^H \underline{w}_{jl} \cdot o_j \\
&= \textcolor{blue}{\sum_{l=1}^L \underline{\delta}_l \cdot \underline{w}_{hl}}
\end{align*}

\framebox{$\delta_h = \textcolor{blue}{\sum_{l=1}^L \left( \underline{\delta}_l \cdot \underline{w}_{hl} \right)} \cdot \textcolor{orange}{f'\left( net_h \right)}$}

\framebox{$ \vartriangle w_{kh} = \eta \cdot \delta_h \cdot \tilde{o}_k $}

\item[]
\underline{Kontakt:}
\\
hendriks@cs.uni-bonn.de

\end{itemize}

\end{document}
